\documentclass[a4paper, 12pt]{article}

% --- PACKAGES ESSENTIELS ---
\usepackage[utf8]{inputenc}      % Encodage des caractères
\usepackage[T1]{fontenc}         % Encodage de la police
\usepackage[english]{babel}      % Langue anglaise (césure correcte)
\usepackage{graphicx}            % Pour les images
\usepackage{float}               % Pour forcer le positionnement des images (H)
\usepackage{subcaption}

% --- MISE EN PAGE & MARGES ---
\usepackage[margin=2.5cm]{geometry} % Marges standards de 2.5cm
\usepackage{parskip}             % Ajoute de l'espace entre les paragraphes (plus moderne)
\usepackage{microtype}           % Améliore grandement la justification du texte
\usepackage{chngcntr}
\counterwithin{figure}{section}

% --- LISTES & TITRES ---
\usepackage{enumitem}            % Pour personnaliser les listes (puces, description)
\usepackage{titlesec}            % Pour le style des titres

% --- LIENS HYPERTEXTES ---
\usepackage{hyperref}
\hypersetup{
    colorlinks=true,
    linkcolor=black,
    urlcolor=blue,
    pdftitle={Data Visualization Report}
}

\begin{document}

% =========================
% Title Page
% =========================
\begin{titlepage}
    \thispagestyle{empty}
    
    \noindent
    \begin{minipage}{0.45\textwidth}
        \includegraphics[width=0.6\linewidth]{title_page/upm_logo.png}
    \end{minipage}%
    \hfill
    \begin{minipage}{0.45\textwidth}
        \raggedleft
        \includegraphics[width=0.6\linewidth]{title_page/etsi_logo.png}
    \end{minipage}
    
    \vspace{1cm}

    \begin{center}

        {\Large \textbf{Universidad Politécnica de Madrid}}\\[0.3cm]
        {\large Escuela Técnica Superior de Ingenieros Informáticos}\\[2cm]
    
    \end{center}

    
    \begin{center}

        
        {\Large Data Visualization}\\[0.5cm]
        
        {\LARGE \textbf{World Bank Project : An analysis of Global Connectivity, Wealth, and Education}}\\[0.3cm] % Ajoute de l'espace ici
    \end{center}
    
    \vfill
    \begin{flushleft}
        Carlos Manzano Izquierdo\\
        Guillermo Bermejo Babiano\\
        Stefano Longo\\
        Melen Laclais\\[0.4cm]
        \texttt{carlos.manzano@alumnos.upm.es}\\
        \texttt{g.bbabiano@alumnos.upm.es}\\
        \texttt{stefano.longo@alumnos.upm.es}\\
        \texttt{melen.laclais@alumnos.upm.es}\\
    \end{flushleft}
    
    \vspace{0.8cm}
    \begin{center}
        Madrid, \today
    \end{center}
\end{titlepage}

% =========================
% Table of Contents Page
% =========================
\newpage
\pagenumbering{arabic} % Numérotation romaine pour le sommaire
\tableofcontents
\newpage

% =========================
% Main Content
% =========================

\section{Introduction}

\newpage
\section{Dataset}

\newpage
\section{Design abstraction}

\newpage
\section{How has global Internet access evolved over time?}

\subsection{Problem characterization}

\subsection{Data and task abstraction}

\subsection{Interaction and visual encoding}

\subsection{Algorithmic implementation}

\subsection{Results}

\newpage
\section{What are the potential correlations and dependencies between Internet access, GDP per capita, and other socioeconomic indicators?} \label{GDP_Internet}

\subsection{Problem characterization}

The rapid expansion of the digital economy has made Internet connectivity a critical determinant of modern economic development. However, the relationship between connectivity and socioeconomic progress is rarely linear or uniform across different global regions. To understand the "Digital Divide," it is insufficient to look at Internet penetration in isolation; it must be contextualized against economic output and social indicators.

The domain-specific challenge addressed in this section is the complexity of analyzing multi-variate relationships that evolve over time. Policy analysts and development economists often face the following obstacles when working with raw World Bank data:

\begin{itemize}
    \item \textbf{Multidimensionality:} Users need to simultaneously assess the interplay between a country's wealth (GDP per Capita), its social structure (e.g., Female Employment), and its technological adoption (Internet Usage), while also considering population size and geographic region.
    \item \textbf{Temporal Evolution:} A static snapshot of a single year fails to capture the \textit{trajectory} of development. It is crucial to identify whether economic growth precedes internet adoption or vice versa, and how these trends have shifted from 2000 to 2020.
    \item \textbf{Hypothesis Generation:} As outlined in the project objectives, the path to a solution is not predetermined. Users do not always know the exact questions to ask in advance. Therefore, the goal is not merely to present a summary statistic, but to support \textit{exploratory analysis} that allows users to discover potential correlations, clusters, and outliers dynamically.
\end{itemize}

Consequently, the design problem is characterized by the need to augment the user's ability to perceive correlations and dependencies among these continuous variables without being overwhelmed by the volume of longitudinal data. The system must facilitate the transition from a general global overview to specific country-level insights, supporting the cognitive task of linking economic indicators to digital adoption rates.

% Skeleton for an optional conceptual image (e.g., a diagram showing the relationship between the variables or the hypothesis workflow)
% \begin{figure}[H]
%     \centering
%     \includegraphics[width=0.8\textwidth]{path/to/your/conceptual_diagram.png}
%     \caption{Conceptual model of the dependencies between Socioeconomic factors and Internet connectivity.}
%     \label{fig:q2_problem_char}
% \end{figure}

\subsection{Data and task abstraction}

To translate the domain characterization into actionable design requirements, we analyze the dataset structure and abstract the user's goals into generic visualization tasks.

\subsubsection*{Data Abstraction}
The underlying dataset is a table of quantitative and categorical indicators derived from the World Bank, structured as follows:

\begin{itemize}
    \item \textbf{Items:} The primary items are countries, observed annually.
    \item \textbf{Attributes:}
    \begin{itemize}
        \item \textbf{Categorical:} \textit{Country Name} and \textit{Region} serve as identity channels and grouping factors.
        \item \textbf{Quantitative (Sequential):} \textit{Internet Usage (\%)} and \textit{Female Employment (\%)} are ratio variables bounded between 0 and 100.
        \item \textbf{Quantitative (Diverging/Sequential):} \textit{GDP per Capita} is a ratio variable with a high dynamic range, necessitating a derived attribute \textit{Log\_GDP} to handle skewness. \textit{Population} is used to weight the visual elements.
        \item \textbf{Ordinal:} \textit{Year} (2000–2020) orders the data chronologically, allowing for temporal analysis.
    \end{itemize}
\end{itemize}

\subsubsection*{Task Abstraction}
Following the design framework, the user's interaction with the system is mapped to the "Why, What, and How" structure:

\begin{description}
    \item[Why (Goal): Discover and Hypothesize.] \hfill \\
    The primary goal is \textbf{exploratory analysis}. The user does not start with a specific lookup query but intends to \textit{discover} new patterns. The visualization must support the generation of hypotheses regarding the causality or correlation between economic power (GDP) and digital adoption.
    
    \item[What (Target): Trends, Correlations, and Outliers.] \hfill \\
    The user seeks to identify:
    \begin{itemize}
        \item \textit{Correlations:} Is there a linear or logarithmic relationship between wealth and connectivity?
        \item \textit{Trends:} How has the position of specific regions shifted over the 20-year period?
        \item \textit{Outliers:} Which countries deviate significantly from their regional trend (e.g., high GDP but low Internet access)?
    \end{itemize}
    
    \item[How (Method): Filter, Select, and Encode.] \hfill \\
    To support these targets, the system requires:
    \begin{itemize}
        \item \textit{Filtering:} Reducing the complexity by viewing one \textit{Year} at a time to avoid occlusion.
        \item \textit{Selection:} Allowing the user to dynamically \textit{change} the variable mapped to the Y-axis (GDP or Female Employment) to test different dependency models.
        \item \textit{Encoding:} Mapping multidimensional data simultaneously—using position for the primary correlation, size for population context, and color for regional clustering.
    \end{itemize}
\end{description}

\subsection{Interaction and visual encoding}

To address the high dimensionality of the data while maintaining interpretability, we selected the \textbf{Bubble Scatterplot}. This idiom serves as a prime example of "layering" visual channels: it starts with a standard scatterplot to show correlation and adds size and color to encode two additional attributes without requiring a 3D representation.

\subsubsection*{Visual Encoding: Marks and Channels}
Following the "Building Blocks" framework, the visualization is constructed using \textbf{Points} (0D marks) to represent individual countries. The visual channels were mapped based on perceptual effectiveness:

\begin{itemize}
    \item \textbf{Position (X and Y Axes):} The two most critical quantitative attributes—\textit{Internet Usage} and the selected socioeconomic indicator (e.g., \textit{GDP per Capita})—are mapped to spatial position. 
    \begin{itemize}
        \item \textit{Justification:} Spatial position is the most accurate channel for the human perceptual system to estimate magnitude and correlation.
    \end{itemize}
    
    \item \textbf{Size (Area):} The \textit{Population} of each country determines the area of the point. 
    \begin{itemize}
        \item \textit{Justification:} This adds necessary weight to the data, allowing the user to distinguish between trends driven by massive economies (e.g., China, India) versus smaller nations. We utilize area rather than radius to ensure the visual magnitude scales linearly with the data.
    \end{itemize}
    
    \item \textbf{Color (Hue):} A categorical palette distinguishes the \textit{Region}. 
    \begin{itemize}
        \item \textit{Justification:} As noted in the design literature, hue is effective for nominal data (identity). We applied opacity (\texttt{alpha = 0.7}) to the marks to mitigate occlusion where countries cluster tightly (e.g., in the lower-left quadrant).
    \end{itemize}
\end{itemize}

% Placeholder for the static view of the encoding
\begin{figure}[H]
    \centering
     \includegraphics[width=0.6\textwidth]{Vis_2/corr_internet_access.png}
    \caption{Visual Encoding strategy: Mapping Position, Size, and Color to distinct data attributes.}
    \label{fig:q2_encoding}
\end{figure}

\subsubsection*{Interaction: Managing Complexity}
The dataset covers a 20-year span for over 150 countries, presenting a challenge of high information density. Displaying this simultaneously (e.g., via "Small Multiples") would result in excessive visual clutter and cognitive overload. Instead, we adopted a combination of \textbf{"Manipulation"} and \textbf{"Reduction"} strategies to manage this complexity effectively:

\begin{enumerate}
    \item \textbf{Temporal Animation (The "Play" Button):} 
    By automating the \textit{Year} selection, the visualization becomes a movie. This allows the user to perceive the "trajectory" of development—specifically, the global migration from low-income/low-connectivity to high-income/high-connectivity—without burdening working memory with static snapshots.
    
    \item \textbf{Attribute Reconfiguration (Y-Axis Selector):}
    To address the multidimensional nature of the data without crowding the screen with extra axes, we implemented a dropdown control.
    \begin{itemize}
        \item \textit{Implementation:} This allows the user to dynamically swap the dependent variable (e.g., changing from \textit{GDP per Capita} to \textit{Female Employment}).
        \item \textit{Justification:} This strategy enables the user to test different hypotheses (e.g., "Is Internet access more strongly correlated with wealth or gender equality?") while maintaining the same familiar visual metaphor.
    \end{itemize}

    \item \textbf{Categorical Filtering (Region Selector):}
    Given the high number of overlapping points (occlusion), particularly in the lower quadrants of the chart, a "Reduction" strategy was necessary.
    \begin{itemize}
        \item \textit{Implementation:} A multi-select picker allows users to filter the dataset by \textit{Region}.
        \item \textit{Justification:} This allows users to isolate specific geographic clusters (e.g., comparing only \textit{Europe} vs. \textit{Sub-Saharan Africa}), thereby reducing visual noise and making local patterns distinct.
    \end{itemize}
    
    \item \textbf{Geometric Transformation (Logarithmic Scale):} 
    The distribution of GDP per Capita follows a power law, causing meaningful differences between developing nations to be compressed into a small visual space. 
    \begin{itemize}
        \item \textit{Implementation:} A toggle switch transforms the Y-axis to a Logarithmic scale ($\log_{10}$). This spreads out the lower values, making the correlation visible across all income levels, not just wealthy nations.
    \end{itemize}
    
    \item \textbf{Details-on-Demand:} 
    While size and color are effective for overview, they lack precision. A hover interaction provides a tooltip with exact numerical values (Internet \%, exact GDP), bridging the gap between the "Overview" and specific "Details."
\end{enumerate}

% Interaction controls image
\begin{figure}[H]
    \centering
     \includegraphics[width=0.4\textwidth]{Vis_2/image.png}
    \caption{Interaction Controls: The sidebar enables "Reduction" via region filtering and "Manipulation" via variable selection and geometric transformation.}
    \label{fig:q2_controls}
\end{figure}


\subsection{Algorithmic implementation}

The visualization is implemented as an encapsulated R Shiny module (\texttt{vis2\_ui} and \texttt{vis2\_server}), ensuring modularity and preventing namespace collisions with other sections of the application. The rendering pipeline consists of three distinct stages: reactive data processing, aesthetic mapping, and interactive conversion.

\subsubsection*{Reactive Data Processing}
To ensure low-latency updates during the animation loop, data manipulation is handled within a \texttt{reactive(\{\})} context. The algorithm performs the following operations whenever user inputs change:

\begin{enumerate}
    \item \textbf{Input Listening:} The system monitors three input vectors: the temporal trigger (\texttt{year\_select}), the categorical filter (\texttt{region\_select}), and the variable selector (\texttt{y\_var}).
    \item \textbf{Subset Operation:} The global World Bank dataset is filtered using \texttt{dplyr} logic. 
    \begin{itemize}
        \item \textit{Temporal:} \texttt{filter(Year == input\$year\_select)} reduces the dataset to a single time slice (~180 rows), significantly reducing the rendering overhead compared to processing the full longitudinal dataset.
        \item \textit{Categorical:} A vector comparison filters countries based on the selected regions.
    \end{itemize}
    \item \textbf{Conditional Variable Selection:} A conditional control structure handles the geometric transformation logic. If the "Log Scale" switch is active (\texttt{input\$log\_scale == TRUE}) and the selected variable is GDP, the system dynamically swaps the mapping to the pre-calculated \texttt{Log\_GDP\_Per\_Capita} column to ensure correct axis scaling.
\end{enumerate}

\subsubsection*{Rendering Pipeline (ggplot2 + Plotly)}
The visualization employs a hybrid rendering approach. The static graphical grammar is defined using \texttt{ggplot2} for its robust layering system, while the interactivity is injected via the \texttt{plotly} library.

\begin{itemize}
    \item \textbf{Aesthetic Mapping:} The \texttt{aes()} function maps the filtered data to the visual channels defined in Section 5.3:
    \begin{verbatim}
    aes(x = Internet_Perc, y = !!y_sym, size = Population, color = Region)
    \end{verbatim}
    Note the use of the bang-bang operator (\texttt{!!}) to programmatically evaluate the user-selected Y-variable symbol.
    
    \item \textbf{Custom Tooltip Generation:} To provide the "Details-on-Demand" identified in the task abstraction, the standard hover behavior is overridden. A custom HTML string is constructed within the aesthetic mapping using \texttt{paste()}. This string concatenates Country Name, Region, Year, and formatted numerical values, which is then passed to the \texttt{text} aesthetic.
    
    \item \textbf{Interactive Conversion:} The final object is rendered using \texttt{ggplotly(p, tooltip = "text")}. This wrapper translates the R graphical objects into a JavaScript-based D3.js chart, enabling client-side interactions such as zooming, panning, and hovering without requiring a round-trip to the server.
\end{itemize}
\subsection{Results}

The interactive nature of the \textit{Dynamic Bubble Scatterplot} allowed for the validation of several hypotheses regarding the "Digital Divide." By manipulating the temporal and geometric controls, three distinct patterns emerged:

\subsubsection*{1. The Economic Driver: Wealth as a Predictor of Connectivity}
When mapping \textit{GDP per Capita} against \textit{Internet Usage}, a robust positive correlation is observed globally. 
\begin{itemize}
    \item \textbf{Linearity in Log-Scale:} While the relationship appears exponential on a linear scale, applying the logarithmic transformation reveals a strong linear dependency. This suggests that for every order of magnitude increase in wealth, there is a consistent proportional increase in digital adoption.
    \item \textbf{Global Convergence:} The animation reveals that while the correlation remains strong, the "entry threshold" has lowered. In 2000, only wealthy nations had significant access; by 2020, even lower-middle-income nations show rapid acceleration in internet adoption, although the wealth gap remains distinct.
\end{itemize}

\begin{figure}[H]
    \centering
    % PLACEHOLDER: Insert image of GDP vs Internet (Log Scale) showing the linear trend
    \includegraphics[width=0.8\textwidth]{Vis_2/gdp.png}
    \caption{GDP per Capita (Log Scale) vs. Internet Usage (2020). A clear linear trend demonstrates that economic output is the primary predictor of connectivity.}
    \label{fig:results_gdp}
\end{figure}

\subsubsection*{2. The Social Divergence: Female Employment Patterns}
Unlike GDP, the correlation between \textit{Female Employment} and \textit{Internet Usage} is not universally linear; it is highly dependent on the \textit{Region}, a pattern revealed by using the Region Filter control.

\begin{itemize}
    \item \textbf{Europe, Central Asia, and MENA:} In these regions, a positive linear relationship is observable. As internet penetration increases, female participation in the workforce tends to rise, suggesting that digital economies may facilitate inclusive employment.
    \item \textbf{Sub-Saharan Africa:} This region exhibits a distinct, sparse pattern. High variation in female employment exists regardless of internet connectivity. In many agrarian economies within this region, female employment is high (in agriculture) even with low internet access. Conversely, other nations show low participation despite growing connectivity. This non-linearity likely reflects complex structural, cultural, or religious constraints that cannot be overcome by digital infrastructure alone.
\end{itemize}

\begin{figure}[H]
    \centering
    % PLACEHOLDER: Insert image comparing Europe vs Sub-Saharan Africa patterns
    \includegraphics[width=0.8\textwidth]{Vis_2/female.png}
    \caption{Female Employment vs. Internet Usage. The linear trend visible in Europe (left/top points) contrasts sharply with the scattered distribution in Sub-Saharan Africa, highlighting regional socio-cultural differences.}
    \label{fig:results_female}
\end{figure}

\subsubsection*{3. Temporal Velocity and "Leapfrogging"}
The animation feature revealed a key temporal insight: the speed of adoption is increasing. 
\begin{itemize}
    \item \textbf{Trajectory:} Developing nations in Asia and Latin America are moving along the X-axis (Internet adoption) significantly faster than Western nations did during the early 2000s. This "leapfrogging" effect is visible as bubbles migrate rapidly to the right side of the chart within the last 5 years of the timeline.
\end{itemize}

\subsubsection*{Conclusion on Design Effectiveness}
These insights confirm the effectiveness of the design choices tailored to the problem characterization. 
\begin{itemize}
    \item The \textbf{Log-Scale Toggle} was essential for identifying the linear economic correlation that was otherwise hidden by the power-law distribution of wealth.
    \item The \textbf{Region Filter} (Reduction strategy) was critical in disaggregating the global data to reveal that Female Employment trends are regionally specific rather than universally applicable.
    \item The \textbf{Animation} successfully transformed a static dataset into a narrative of development, allowing us to observe the acceleration of the digital economy over two decades.
\end{itemize}

\newpage
\section{Global Comparison of Digital and Educational Access}

\subsection{Problem Characterization}

In the context of global development, the "Digital Divide" and educational inequality are two major challenges of the last decade. While the Internet has become a fundamental tool for economic and social participation, its adoption relies heavily on foundational skills, primarily literacy. However, raw datasets from organizations like the World Bank are often fragmented or vast, making it difficult for policymakers, researchers, and the general public to intuitively understand the relationship between a population’s ability to read (Literacy Rate) and their access to digital infrastructure (Internet Access).

The core problem this visualization addresses is the complexity of analyzing these two development indicators simultaneously across space and time. A simple tabular view does not reveal whether the digital gap is closing in developing regions like Sub-Saharan Africa or South Asia, nor does it highlight if high literacy is a guaranteed predictor of high internet adoption. Furthermore, regional disparities are often masked by global averages, requiring a tool that allows for granular analysis at the continent and country levels.

Therefore, the objective of this analysis is to provide an interactive interface to explore the spatiotemporal evolution of these indicators. The visualization aims to answer the following key questions:

% Utilisation de l'environnement 'description' pour tes questions clés
\begin{enumerate}
    \item \textit{How are literacy rates and internet access distributed geographically, and where are the major disparities located?}

    \item \textit{Is there a positive correlation between a country's literacy rate and its internet penetration, and does this relationship hold true across all continents?}

    \item \textit{How have these indicators evolved over the last two decades (2000–2022), and which regions are catching up the fastest?}
\end{enumerate}    
    
\subsection{Data and Task Abstraction}

The dataset consists of two primary World Bank indicators merged into a single tabular structure. The raw data was originally in a "wide" format (with years 1960–2024 as columns) and was transformed into a "long" format to facilitate time-series rendering. The primary items are countries, identified by their ISO-3 codes. The quantitative attributes mapped to visual channels are Internet Access (\% of population) and Adult Literacy Rate (\% of people ages 15 and above). These are analyzed against the sequential attribute of Year (ranging from 2000 to 2022) and the categorical attribute of Region/Continent, derived using the \texttt{countrycode} library to allow for hierarchical filtering. 

A significant challenge was the sparsity of the literacy dataset, as surveys are not conducted annually. To address this, we derived a continuous dataset using a the last known value forward to subsequent years, ensuring continuous trends without artificial gaps. Additionally, we computed aggregated metrics, specifically World and Region Averages, to serve as context layers for the time-series analysis.

The visualisation is designed to support several domain-specific objectives. First, it allows for correlation analysis to verify whether education level is a prerequisite for digital adoption, identifying whether countries align on a positive correlation diagonal or form distinct clusters based on their level of development, and adapts according to region. It can also compare spatial-temporal trends to track the evolution of specific countries over time, determining whether a selected country is “catching up” with the global average or falling behind. Thirdly, it summarises spatial distributions to identify continents that consistently perform well or poorly on both indicators, providing a better understanding of the digital divide. Finally, the interface facilitates ranking and search tasks, allowing users to retrieve exact values for specific items by hovering over them and to identify the best-performing countries in a selected region.

\subsection{Interaction and Visual Encoding}

We follow the Shneiderman's mantra: "Overview first, zoom and filter, then details-on-demand." The interface is structured to prioritize immediate comparison while minimizing cognitive load and scrolling.

\begin{figure}[H]
    \centering
    \includegraphics[width=0.6\textwidth]{Vis_3/dashboard_layout.png}
    \caption{Global layout showing the sidebar controls (left) and the side-by-side maps (top), allowing for direct comparison without scrolling.}
    \label{fig:layout}
\end{figure}

\subsubsection*{Layout and Spatial Arrangement}
To ensure a comprehensive view of the data without requiring the user to scroll, we adopted a sidebar layout for the control panel on the left. Placing parameters horizontally at the top would have pushed the analytical charts below the "fold," disrupting the user's workflow. The current layout guarantees that all four coordinated views, spatial, temporal, and relational, remain visible simultaneously within a single screen.

Regarding the spatial overview, we juxtaposed the two choropleth maps side-by-side rather than using a toggle button to switch between them. A toggle design relies heavily on the user's working memory to recall the previous state, whereas juxtaposition allows for direct, eyes-on comparison. To further enhance this, we implemented a synchronized navigation mechanism: zooming or panning on one map automatically updates the viewpoint of the other. This ensures that the user always compares the exact same geographical extent, facilitating the detection of local discrepancies between digital adoption and literacy.

\begin{figure}[H]
    \centering
    \includegraphics[width=0.6\textwidth]{Vis_3/zoom_interaction.png}
    \caption{Synchronized zoomed view allowing for local comparison, with specific values revealed via tooltips.}
    \label{fig:layout}
\end{figure}

\subsubsection*{Visual Encoding and Color Strategy}
We selected distinct hues to maximize discriminability while maintaining semantic resonance. \textbf{Purple} was chosen for Internet Access, associating the color with the digital and technological infrastructure. \textbf{Green} was selected for Literacy Rate to symbolize human growth and positive social development. This color coding is consistently applied across all visualizations, including the time series lines, to prevent confusion. To aid value estimation, we utilized a discrete 5-step legend to create facility for the user to recognize the value of a color. 

For the correlation scatter plot, we employed the \textbf{Viridis} color scale (transitioning from purple to green/yellow), it bridges the two primary themes and provides a perceptually uniform scale that remains legible even for color-blind users.

\begin{figure}[H]
    \centering
    \includegraphics[width=0.4\textwidth]{Vis_3/correlation.png}
    \caption{Viridis color scale on the correlation scatter plot.}
    \label{fig:layout}
\end{figure}


\subsubsection*{Temporal Interaction and Data Handling}
The temporal exploration is driven by a slider that allows users to select the year of analysis. This enables the observation of "live" evolution, particularly the rapid global spread of internet access. We handled the missing data: if a country lacks a survey value for the specific selected year, the system automatically retrieves and displays the last available value. This also prevents the map polygons from flashing empty during animation. The interface remains transparent about this mechanism via tooltips, which explicitly state the actual year of the data point or "\textit{No data}" if the country don't have literacy data. (e.g., \textit{"Literacy: 95\% (2015)"}). A play button also allows you to view an animation showing the gradual changes on both maps, which also changes the values of the other graphs accordingly.

\begin{figure}[H]
    \centering
    \includegraphics[width=0.4\textwidth]{Vis_3/temporal_interaction.png}
    \caption{Temporal evolution controls.}
    \label{fig:temporal}
\end{figure}

We also add a red time bar to the Time Series chart to remind users which year they have selected in the slider. This allows the users to don't look again for which year they choose.

\begin{figure}[H]
    \centering
    \includegraphics[width=0.4\textwidth]{Vis_3/temporal_interaction_time_serie.png}
    \caption{The time series chart shows the selected year with a vertical red line marker.}
    \label{fig:temporal}
\end{figure}

\subsubsection*{Contextualization and Details-on-Demand}
The dashboard offers several layers of detail to support in-depth analysis. The "Evolution Over Time" chart contextualizes the selected country's performance by allowing the overlay of World and Region averages. These benchmarks are rendered as dotted lines to distinguish them from the solid country trend.

\begin{figure}[H]
    \centering
    \includegraphics[width=0.2\textwidth]{Vis_3/tooltip_evolution_over_time.png}
    \caption{Evolution over time tooltip.}
    \label{fig:temporal}
\end{figure}

To support "details-on-demand," hovering over a country updates a specific "Hover Details" bar chart in the sidebar. For this chart, we opted for a neutral color palette to focus the user's attention purely on the proportional difference between the two metrics, unbiased by the thematic colors. Additionally, a dynamic "Top Ranking" list provides a textual summary of the top 5 performers in the active region, enabling users to rapidly identify top-performing nations within the selected region and verify if high internet adoption aligns with educational leadership.

\begin{figure}[H]
    \centering
    \includegraphics[width=0.2\textwidth]{Vis_3/hover_details.png}
    \caption{Hover details for direct comparison.}
    \label{fig:temporal}
\end{figure}

\begin{figure}[H]
    \centering
    \includegraphics[width=0.2\textwidth]{Vis_3/ranking.png}
    \caption{Ranking according to selected region.}
    \label{fig:temporal}
\end{figure}

Finally, the "Region Focus" selector acts as a global filter. By narrowing the scope to a specific continent (e.g., \textit{Africa}), the system zooms the maps, filters the scatter plot to reduce overplotting, and updates the ranking list. This interaction allows users to shift from a global perspective to a local analysis, revealing correlations that might be masked in the dense global dataset.

\begin{figure}[H]
    \centering
    \includegraphics[width=0.6\textwidth]{Vis_3/africa_example.png}
    \caption{Data reduction through regional filtering: Focusing on Africa to eliminate global overplotting and reveal local correlation patterns.}    \label{fig:temporal}
\end{figure}

\subsubsection*{Country and Region Search}

To facilitate the identification of specific countries within the global dataset, we integrated a search bar based on the \texttt{selectize} library. This component allows for rapid filtering and direct localization (with a automatic zoom on the country selected) and adapts according to the region selected.

\begin{figure}[H]
    \centering
    \includegraphics[width=0.2\textwidth]{Vis_3/search_bar_capture.png}
    \caption{The search bar : users can select a country to trigger an focus and synchronization across the others dashboard components.}
    \label{fig:search_bar}
\end{figure}

\subsection{Algorithmic Implementation}

The technical implementation of the dashboard leverages the Plotly library for high-performance reactive visualizations. The system is architected around a centralized state management model to ensure consistency across all coordinated views.

\subsubsection*{Synchronized Geospatial Proxy}
One of the primary technical challenges was the bi-directional synchronization of the two choropleth maps. To avoid infinite reactive loops where Map A updates Map B, which in turn tries to update Map A, we implemented a \textbf{Plotly Proxy} combined with a mutual exclusion flag (\textit{syncing state}).
\begin{itemize}
    \item When a user zooms or pans on a map, the \texttt{plotly\_relayout} event captures the new coordinate bounds (longitude, latitude, and projection scale).
    \item A reactive observer filters these events and stores the new state in a \texttt{reactiveValues} object.
    \item The update is then pushed to the twin map using the \texttt{plotlyProxyInvoke} method, which allows for smooth visual updates without re-rendering the entire graphical object, significantly improving the frame rate.
\end{itemize}

\subsubsection*{Hierarchical Filtering}
To support local analysis, the system implements a hierarchical filtering algorithm. Based on the user's "Region Focus" selection, the system performs a lookup against a custom reference table (\textit{world\_ref}) derived from the \texttt{rnaturalearth} package. This mapping translates continental categories into specific lists of ISO-3 country codes. These codes are then used to prune the global dataset before it reaches the rendering functions of the scatter plot and ranking lists. This "pre-filtering" step reduces the number of SVG elements rendered in the browser, ensuring that even with dense datasets, the correlation analysis remains responsive to user interactions.

\subsubsection*{Coordinated Interaction Logic}
The linking between views is managed through a global \texttt{selection\$code} reactive variable. When a country is clicked on a map, the system triggers a cascade of updates:
\begin{enumerate}
    \item The \textbf{Time Series} chart executes a specific query to the full historical dataset to render the longitudinal trend.
    \item The \textbf{Scatter Plot} re-calculates the size and opacity channels for all points, highlighting the selected item using a conditional logic (\texttt{ifelse(is\_selected, 15, 8)}).
    \item The \textbf{Hover Details} bar chart utilizes a dedicated observer that monitors the mouse position (\texttt{plotly\_hover}) to provide real-time, low-latency numerical feedback in the sidebar.
\end{enumerate}

\subsection{Results}

The interactive nature of the dashboard allowed for several key insights into the relationship between education and digital development. By manipulating the temporal, spatial, and relational views, we identified three major trends.

\subsubsection*{The Educational Threshold for Digital Adoption}
As shown in the 2022 global snapshot (Figure \ref{fig:global_corr}), there is a clear threshold effect. The scatter plot analysis reveals that no country achieves more than 50\% internet penetration without first surpassing an 80\% literacy rate. This confirms our initial hypothesis that basic education is a non-negotiable prerequisite for the digital transition.

\begin{figure}[H]
    \centering
    \includegraphics[width=0.3\textwidth]{Vis_3/results_correlation.png}
    \caption{Global correlation in 2022 showing the clustering of developed nations and the minimum literacy threshold required for digital growth.}
    \label{fig:global_corr}
\end{figure}.

By animating the data from 2000 to 2022, we observed a recurring pattern in emerging economies: a significant temporal gap between the attainment of high literacy and the surge in digital connectivity. Contrary to the simplistic view that literacy growth linearly drives internet adoption, the data reveals that literacy often reaches a stable plateau long before any significant digital growth occurs. 

We characterize this as \textbf{Latent Potential}. Literacy acts as a ``structural floor'', a necessary cognitive foundation that remains dormant until an external catalyst, such as infrastructure deployment or market liberalization, triggers an explosion in access.

Vietnam serves as a primary example of this trajectory. By selecting Vietnam in the time-series module, we observe that the country had already secured a literacy rate of over 90\% in the early 2000s. However, internet penetration remained below 15\% for several years. This 15-year gap highlights that the population was ``ready to connect'' (educationally prepared) but was waiting for infrastructural catalysts. Once fiber optics and mobile data became affordable post-2010, the country experienced a vertical surge, reaching over 70\% by 2022. The dashboard clearly visualizes this ``leap'' from a stable educational base.

\begin{figure}[H]
    \centering
    \includegraphics[width=0.4\textwidth]{Vis_3/vietnam_latent_potential.png}
    \caption{Analysis of Vietnam's trajectory: The near-universal literacy rate (green), achieved decades ago, represents a massive ``latent potential'' that was only activated after 2005 with the modernization of telecommunications infrastructure.}
    \label{fig:vietnam_latent_potential}
\end{figure}

\subsubsection*{Case Study: India and the Infrastructure Trigger}
A similar pattern is observed in India, but with a more localized trigger. The literacy rate shows steady, incremental growth, while internet access remained largely stagnant until 2015. After this point, the digital curve displays an almost vertical explosion. This observation proves that while the educational floor was rising, the actual explosion was decoupled from literacy and was instead triggered by external economic factors, such as the entry of low-cost telecommunication operators and the mass adoption of smartphones.

\begin{figure}[H]
    \centering
    \includegraphics[width=0.4\textwidth]{Vis_3/india_latent_potential.png}
    \caption{Time-series analysis of India : The visual ``lag'' between stable literacy (green) and the sudden exponential growth of internet access (purple).}
    \label{fig:lag_analysis}
\end{figure}

\subsubsection*{The ``Development Trap'' in Low-Literacy Regions}
To validate the necessity of the educational prerequisite, we utilized the dashboard to analyze countries with low literacy rates ($<40\%$), such as Niger or South Sudan. In these instances, even when infrastructure is introduced, the digital curve fails to follow the same exponential trajectory seen in Vietnam. This confirms a critical insight: \textbf{infrastructure without literacy creates a ceiling.} A population that cannot read or write cannot easily bypass the barriers of a text-heavy digital world, regardless of how cheap the data becomes. This ``Dual Barrier'' (low education + low infrastructure) is clearly visible when comparing these regions with the World Average using the overlay feature.

\begin{figure}[H]
    \centering
    \begin{minipage}{0.4\textwidth}
        \centering
        \includegraphics[width=\linewidth]{Vis_3/niger_dual_maps_2005.png}
        \footnotesize{\textbf{(a) Spatial Distribution in 2005: Internet (left) vs. Literacy (right)}}
    \end{minipage}
    \begin{minipage}{0.4\textwidth}
        \centering
        \includegraphics[width=\linewidth]{Vis_3/niger_timeseries_evolution.png}
        \footnotesize{\textbf{(b) Longitudinal Evolution (2000-2022)}}
    \end{minipage}

    \caption{ The juxtaposition in 2005 \textbf{(a)} shows a critical lack of educational foundations. The subsequent time-series \textbf{(b)} demonstrates that despite global technological progress, internet adoption remains stagnant, failing to cross the ``literacy ceiling'' required for an exponential surge.}
    \label{fig:niger_analysis}
\end{figure}

In conclusion, while Literacy and Internet Access are highly correlated in a static snapshot, their temporal relationship is characterized by a complex dynamic of \textbf{preparation and activation}. Our findings suggest that literacy acts as a foundational prerequisite—a structural floor—while infrastructure serves as the catalyst that triggers the digital surge.

However, it remains difficult to claim a direct causality. The simultaneous growth of these two indicators is likely influenced by a latent third factor, such as national economic wealth (GDP) or institutional stability, which provides the necessary resources for both educational and technological investment.

\subsubsection*{Regional Analysis of the Digital Divide}
The 2022 analysis reveals that global progress hasn't erased regional inequalities. While Europe has reached a 'saturation point' with near-universal literacy and digital access, Africa shows extreme dispersion. In many African nations, identical literacy rates lead to vastly different digital outcomes, proving that education alone cannot overcome the barriers of high infrastructure costs and economic instability.

\begin{figure}[H]
    \centering
    % --- WORLD ---
    \begin{minipage}{0.32\textwidth}
        \centering
        \includegraphics[width=\linewidth]{Vis_3/world_correlation_2022.png}
        \vspace{0.1cm}
        \footnotesize{\textbf{(a) Global Correlation}}
    \end{minipage}
    \hfill
    % --- EUROPE ---
    \begin{minipage}{0.32\textwidth}
        \centering
        \includegraphics[width=\linewidth]{Vis_3/europe_correlation_2022.png}
        \vspace{0.1cm}
        \footnotesize{\textbf{(b) Europe (Maturity)}}
    \end{minipage}
    \hfill
    % --- AFRICA ---
    \begin{minipage}{0.32\textwidth}
        \centering
        \includegraphics[width=\linewidth]{Vis_3/africa_correlation_2022.png}
        \vspace{0.1cm}
        \footnotesize{\textbf{(c) Africa (High Disparity)}}
    \end{minipage}

    \caption{Comparative analysis of regional disparities in 2022. The global view \textbf{(a)} masks striking contrasts between European technological saturation \textbf{(b)} and the high heterogeneity of the African continent \textbf{(c)}, where literacy does not yet guarantee equitable digital access.}
    \label{fig:regional_correlations_2022}
\end{figure}


%%%%%%%%%%%%%%%%%%%%%%%%%%%%%%%%%%%%%%%%%%%%%%%%%%%%%%%%%%%%%
\newpage
\section{Which countries show outlier behavior in Internet access or GDP per capita relative to their regional or income group, and how can these deviations be characterized?}

\subsection{Problem Characterization}

As we saw in Section \ref{GDP_Internet}, there is generally a positive correlation between a nation's wealth (GDP per capita) and its technological maturity (Internet Access). Under normal circumstances, economic power is a reliable predictor of digital infrastructure. However, raw datasets often obscure the most analytically valuable cases: the anomalies. Policy researchers and development economists are frequently less interested in the general trend and more focused on identifying countries that "break the rules" of this correlation.

The core problem this visualization addresses is the difficulty of detecting and characterizing these socioeconomic outliers within a dense global dataset. A standard static scatter plot or tabular report does not immediately reveal which nations are over-performing their economic constraints (high digital adoption despite low GDP) or under-performing relative to their potential. Furthermore, reliance on static political labels, such as the World Bank Income Groups, often masks the true digital reality of a nation. For instance, a "Lower Middle Income" country might exhibit the digital characteristics of a "High Income" nation, but this distinction is lost in traditional categorical filtering.

Therefore, the objective of this analysis is to provide a dynamic, algorithmic interface to automatically detect and classify these deviations. By integrating unsupervised learning (K-Means clustering) directly into the visual exploration, the tool aims to answer the following key questions:

\begin{enumerate}
    \item \textit{Which specific countries exhibit statistically significant outlier behavior where digital adoption does not align with economic wealth?}
    \item \textit{Do data-driven clusters (based on actual performance) align with official World Bank Income Groups, or are there significant discrepancies?}
    \item \textit{Can we identify "role model" nations that have managed to bridge the digital divide despite limited economic resources?}
\end{enumerate}


%%%%%%%%%%%%%%%%%%%%%%%%%%%%%%%%%%%%%%%%%%%%%%%%%%%%%%%%%%%%%
\subsection{Data and Task Abstraction}

To support the algorithmic detection of anomalies, we translated the domain characterization into actionable design requirements. This analysis relies on a specific subset of the primary dataset, transformed through a strict processing pipeline to support unsupervised learning.

\subsubsection{Data Abstraction}
The dataset is structured as a multidimensional table where the primary **Items** are sovereign countries (identified by ISO-3 codes). The attributes are classified and transformed as follows:

\textbf{1. Attributes and Types}
\begin{itemize}
    \item \textbf{Categorical (Identity \& Grouping):}
    \begin{itemize}
        \item \textit{Region:} Used for filtering and context.
        \item \textit{World Bank Income Group:} Used as a "ground truth" label for comparison.
        \item \textbf{Integration Note:} These are attached via metadata integration using the \texttt{countrycode} library. While \textit{Literacy Rate} was considered, it is excluded from the clustering vector space to avoid data loss due to sparsity.
    \end{itemize}

    \item \textbf{Quantitative (Sequential \& Transformed):}
    \begin{itemize}
        \item \textit{Internet Access (\%):} A ratio variable bounded between 0--100.
        \item \textit{Log-GDP (Transformed):} The raw GDP variable spans several orders of magnitude (approx. \$200 to \$100,000+). To normalize this heavy right-skew, we compute the base-10 logarithm: $GDP' = \log_{10}(\max(GDP, 1))$. This transformation compresses the scale, ensuring that high-income differences do not dominate the variance.
        \item \textit{Z-Score (Standardized):} Before clustering, both attributes are scaled to $\mu=0, \sigma=1$ ($z = \frac{x - \mu}{\sigma}$). This is critical to ensure both variables contribute equally to the Euclidean distance calculation.
    \end{itemize}

    \item \textbf{Ordinal (Derived):}
    \begin{itemize}
        \item \textit{Cluster ID:} We execute \textbf{Dynamic K-Means ($k=3$)}. To ensure semantic meaningfulness, the system calculates the "development score" of centroids to dynamically label clusters as \textit{"Low," "Medium,"} and \textit{"High Development,"} creating an ordinal ranking rather than nominal IDs.
        \item \textit{Outlier Flag (Boolean):} Derived by comparing the Euclidean distance against a user-controlled percentile threshold ($p \in [90, 99]$).
    \end{itemize}
\end{itemize}

\textbf{2. Processing Pipeline (Reduction Strategy)}
Unlike time-series visualizations which allow for imputation, the clustering module requires complete data vectors.
\begin{enumerate}
    \item \textbf{Filtering and Joining:} For the selected year, the system performs an \texttt{inner\_join} between GDP and Internet datasets.
    \item \textbf{Strict Filtering Policy:} Any country lacking a recorded value is explicitly removed (\texttt{drop\_na}). This reduction strategy ensures the K-Means algorithm is not biased by imputed or zero-filled values.
\end{enumerate}

\subsubsection{Task Abstraction}
Following the visualization design framework, the user's interaction is mapped to the "Why, What, and How" structure:

\textbf{Why (Goal): Discover and Compare}
\begin{itemize}
    \item \textbf{Discover (Outliers):} The primary goal is to \textit{locate} items that exceed the statistical distance threshold. Users search for "triangular" marks that indicate a breakdown in the standard wealth-connectivity correlation.
    \item \textbf{Compare (Classifications):} Users \textit{compare} the algorithmic \textit{Data-Driven Cluster} against the official \textit{World Bank Income Group} to identify "mismatches" (e.g., a "Lower Middle Income" country appearing in the "High Development" cluster).
\end{itemize}

\textbf{What (Target): Topology and Features}
\begin{itemize}
    \item \textbf{Topology (Anomalies):} The target is not the global trend, but the specific geometric deviations defined by the Euclidean distance of a country's vector from its assigned cluster centroid.
    \item \textbf{Features (Attributes):} The user seeks to identify which specific attribute (GDP or Internet) is driving the deviation.
\end{itemize}

\textbf{How (Method): Encode, Reduce, and Select}
\begin{itemize}
    \item \textbf{Encode:} We map the standardized attributes to position ($x, y$), cluster membership to color, and outlier status to shape (triangle vs. circle) to ensure pre-attentive processing.
    \item \textbf{Reduce:} The system reduces complexity via the strict \texttt{drop\_na} pipeline described above.
    \item \textbf{Select (Details-on-Demand):} Using the coordinated profile view, users select specific points to reveal the Z-scores. This allows characterization (e.g., "This country is an outlier because its Internet Z-score is +2.0 while its GDP Z-score is -0.5").
\end{itemize}


%%%%%%%%%%%%%%%%%%%%%%%%%%%%%%%%%%%%%%%%%%%%%%%%%%%%%%%%%%%%%
\subsection{Interaction and Visual Encoding}

To support the detection and characterization of outliers, we designed a composite view consisting of four coordinated components: a central scatter plot, a parametric control panel, a semantic legend, and a details-on-demand profile view. As shown in the global overview (Figure \ref{fig:global_overview}), we adopted a sidebar layout for the controls to ensure the analytical workspace (the scatter plot) remains the primary focal point without requiring scrolling.

\begin{figure}[h]
    \centering
    \includegraphics[width=1.0\textwidth]{Vis_4/global_view.png}
    \caption{Global Overview of the Outlier Detection Module. The layout is divided into two distinct zones: the Control Panel (left) for parametric adjustments and the Main View (right) for visual analysis. The sidebar also houses the "Details-on-Demand" profile chart, ensuring it does not obscure the data during hover interactions.}
    \label{fig:global_overview}
\end{figure}

\subsubsection{The Dynamic Clustering Scatter Plot}
The central visual component is the scatter plot, which maps the derived attributes to specific visual channels to maximize the discriminability of outliers.

\begin{itemize}
    \item \textbf{Position:} The X-axis represents \textit{GDP per capita}. By default, this is rendered on a logarithmic scale ($\log_{10}$) to compress the long tail of high-income nations and reveal the structure of developing economies. The Y-axis represents \textit{Internet Access} linearly.
    \item \textbf{Color (Cluster Membership):} We utilize a qualitative color palette to encode the \textit{Cluster ID} assigned by the K-Means algorithm. As shown in Figure \ref{fig:scatter_main}, distinct hues distinguish the three development tiers (Green for High, Purple for Medium, Orange for Low).
    \item \textbf{Shape (Outlier Status):} To ensure pre-attentive processing, we utilized a dual-channel encoding for the outlier status. "Normal" countries are rendered as circles. \textbf{Outliers}---defined as those exceeding the distance threshold---are rendered as upward-pointing triangles. This distinct shape allows the user to scan the plot and instantly spot "triangular" deviations among "circular" clusters.
\end{itemize}

\begin{figure}[h]
    \centering
    \includegraphics[width=0.95\textwidth]{Vis_4/scatterplot.png}
    \caption{The K-Means Clustering Scatter Plot (Year 2018). The visualization clearly distinguishes between core cluster members (circles) and statistical outliers (triangles) that drift far from their group's centroid.}
    \label{fig:scatter_main}
\end{figure}

\subsubsection{Parametric Controls and Sensitivity Analysis}
Unlike static visualizations, this dashboard allows the user to manipulate the underlying algorithm in real-time via the sidebar controls (Figure \ref{fig:filtering}).

\begin{itemize}
    \item \textbf{Temporal Slider (2000--2020):} Moving this slider triggers a reactive re-calculation of the K-Means model for the specific year, allowing users to watch outliers emerge over time.
    \item \textbf{Outlier Percentile (Sensitivity):} This slider controls the "strictness" of the anomaly detection. It adjusts the percentile threshold ($90^{th}$ to $99^{th}$) of the distance metric. Lowering this value to 90 reveals more "marginal" outliers, while raising it to 99 isolates only the most extreme cases.
\end{itemize}

\begin{figure}[H]
    \centering
    \includegraphics[width=0.6\textwidth]{Vis_4/filtering.png}
    \caption{Parametric Controls. The user can adjust the temporal scope and the mathematical definition of an "outlier" (percentile threshold) dynamically.}
    \label{fig:filtering}
\end{figure}

\subsubsection{Semantics and Grouping}
To aid interpretation, the legend (Figure \ref{fig:legend}) explicitly maps the combination of Color and Shape. It reinforces that the categorization is hierarchical: a country belongs to a Development Group (Color) and is simultaneously classified as either a Core Member (Circle) or an Outlier (Triangle).

\begin{figure}[H]
    \centering
    \includegraphics[width=0.3\textwidth]{Vis_4/legend.png}
    \caption{Semantic Legend. Explaining the dual encoding of Development Level (Color) and Anomaly Status (Shape).}
    \label{fig:legend}
\end{figure}

\subsubsection{Coordinated Details-on-Demand}
Once an outlier is identified (e.g., a triangle in the Medium Development cluster), the user needs to understand \textit{why} it deviates. We implemented a coordinated interaction model:

\begin{itemize}
    \item \textbf{Tooltip (Identification):} Hovering over a point reveals the exact metadata, including the World Bank classification and the raw Euclidean distance to the centroid (Figure \ref{fig:tooltip}).
    \item \textbf{Z-Score Profile (Characterization):} Simultaneously, the sidebar updates to show a "Country Profile" bar chart (Figure \ref{fig:barchart}). This visualizes the \textbf{Standardized Z-Scores} for both indicators. This is critical for analysis: it allows the user to see if a country is an outlier because of disproportionately high Internet access (positive blue bar) or disproportionately low GDP (negative red bar).
\end{itemize}

\begin{figure}[H]
    \centering
    \begin{subfigure}[b]{0.45\textwidth}
        \centering
        \includegraphics[width=\textwidth]{Vis_4/label.png}
        \caption{Tooltip with raw metrics.}
        \label{fig:tooltip}
    \end{subfigure}
    \hfill
    \begin{subfigure}[b]{0.45\textwidth}
        \centering
        \includegraphics[width=\textwidth]{Vis_4/barchart.png}
        \caption{Z-Score Profile.}
        \label{fig:barchart}
    \end{subfigure}
    \caption{Coordinated Views for Details-on-Demand.  over a country triggers both a metadata tooltip and a statistical profile chart to explain the deviation.}
    \label{fig:details}
\end{figure}

%%%%%%%%%%%%%%%%%%%%%%%%%%%%%%%%%%%%%%%%%%%%%%%%%%%%%%%%%%%%%
\subsection{Algorithmic Implementation}

The technical implementation of the outlier detection module utilizes the R Shiny framework, leveraging the \texttt{stats} package for unsupervised learning and \texttt{plotly} for client-side rendering. The system is architected around a sequential reactive pipeline that ensures the clustering model is recalculated only when specific upstream parameters (Year, Outlier Threshold) are modified.

\textbf{Sequential Reactive Pipeline}
To maintain performance while handling computationally intensive tasks, we separated the data processing into two distinct reactive stages:

\begin{enumerate}
    \item \textbf{Data Preparation (\texttt{data\_year}):} This reactive block handles the "Extract-Transform-Load" (ETL) process. It filters the global dataset by the selected year input and performs the inner joins between GDP and Internet tables. Crucially, it applies the log-transformation ($\log_{10}$) at this stage, creating the \texttt{gdp\_log} variable used by downstream processes.
    \item \textbf{Model Execution (\texttt{clustered}):} This dependent reactive block executes the K-Means algorithm. By isolating this step, the system avoids re-fetching data when the user only changes clustering parameters (like the percentile threshold).
\end{enumerate}

\textbf{Unsupervised Learning and Distance Calculation}
The core logic resides in the \texttt{clustered} reactive function. To ensure valid Euclidean distance calculations, the algorithm performs the following operations:

\begin{itemize}
    \item \textbf{Standardization:} We utilize the \texttt{scale()} function to normalize the input matrix $X$ (Internet, Log-GDP). This centers the data at 0 with a standard deviation of 1.
    \item \textbf{K-Means Execution:} We execute \texttt{kmeans(X, centers=3, nstart=50)}. The \texttt{nstart=50} parameter is critical; it runs the algorithm 50 times with random initializations and selects the result with the lowest within-cluster sum of squares, ensuring a globally optimal solution.
    \item \textbf{Semantic Labeling:} A raw K-Means output assigns arbitrary IDs (1, 2, 3). To map these to meaningful labels ("High", "Medium", "Low"), the algorithm calculates a "Development Score" for each cluster centroid (mean of GDP and Internet). The clusters are then sorted by this score, ensuring that the highest-performing group is consistently labeled "High Development" regardless of the random seed.
    \item \textbf{Anomaly Scoring:} The outlier metric is computed vectorially in R. We calculate the squared Euclidean distance between each observation $x_i$ and its assigned centroid $c_k$:
    \begin{equation}
        d(x_i, c_k) = \sqrt{\sum_{j=1}^{m} (x_{ij} - c_{kj})^2}
    \end{equation}
    In the code, this is implemented as: \texttt{sqrt(rowSums((X - centers)\^{}2))}.
\end{itemize}

\textbf{Event-Driven Coordination}
The synchronization between the scatter plot and the profile chart is managed via the \texttt{event\_data()} observer in Plotly.
\begin{itemize}
    \item We attach the country name to the \texttt{customdata} slot of the scatter plot markers.
    \item When a \texttt{plotly\_hover} event is detected, the observer captures this ID.
    \item This triggers a separate rendering function for the bar chart, which pivots the selected country's data into a long format (\texttt{pivot\_longer}) to generate the Z-score visualization using \texttt{ggplot2}.
\end{itemize}


%%%%%%%%%%%%%%%%%%%%%%%%%%%%%%%%%%%%%%%%%%%%%%%%%%%%%%%%%%%%%
\subsection{Results}

By applying the dynamic K-Means clustering ($k=3$) across the timeline (2014--2018), the visualization revealed distinct patterns of global digital inequality. We identified two major eras of divergence: the "Resource Curse" era of 2014 and the "Digital Decoupling" era of 2018.

\subsubsection{The 2014 Analysis: Divergence and Inequality}
The 2014 snapshot (Figure \ref{fig:2014_general}) highlights the extreme volatility within the "Medium Development" cluster (purple). While the economic indicators (GDP) for these nations are similar, their digital outcomes vary drastically.

\begin{figure}[h]
    \centering
    \includegraphics[width=1.0\textwidth]{Vis_4/Results/2014_General.png}
    \caption{The 2014 Global Landscape. Note the massive vertical dispersion in the Purple (Medium Development) cluster, ranging from near 80\% internet access to less than 20\%.}
    \label{fig:2014_general}
\end{figure}

\textbf{1. The Resource Curse: Azerbaijan vs. Equatorial Guinea}
The most striking finding is the contrast between two oil-rich nations. As shown in Figure \ref{fig:2014_divergence}, both belong to the same economic tier, yet their digital realities are opposites.
\begin{itemize}
    \item \textbf{Azerbaijan (The Leapfrogger):} With a Z-score for Internet of nearly +1.0, it performs like a "High Development" nation.
    \item \textbf{Equatorial Guinea (The Laggard):} Despite being classified as "Upper Middle Income," its Internet Z-score is deeply negative, illustrating that wealth without policy does not create connectivity.
\end{itemize}

\begin{figure}[h]
    \centering
    % Row 1: Azerbaijan
    \begin{subfigure}[b]{0.45\textwidth}
        \centering
        \includegraphics[width=\textwidth]{Vis_4/Results/Azerbaijan_label.png}
        \caption{Azerbaijan Tooltip}
    \end{subfigure}
    \hfill
    \begin{subfigure}[b]{0.45\textwidth}
        \centering
        \includegraphics[width=\textwidth]{Vis_4/Results/Azerbaijan_bar.png}
        \caption{Azerbaijan Z-Scores}
    \end{subfigure}
    
    \vspace{1em} % Space between rows
    
    % Row 2: Equatorial Guinea
    \begin{subfigure}[b]{0.45\textwidth}
        \centering
        \includegraphics[width=\textwidth]{Vis_4/Results/Equatorial_label.png}
        \caption{Eq. Guinea Tooltip}
    \end{subfigure}
    \hfill
    \begin{subfigure}[b]{0.45\textwidth}
        \centering
        \includegraphics[width=\textwidth]{Vis_4/Results/Equatorial_bar.png}
        \caption{Eq. Guinea Z-Scores}
    \end{subfigure}
    
    \caption{The Medium-Development Split (2014). Comparing the "Leapfrogger" (Azerbaijan, Top) with the "Laggard" (Equatorial Guinea, Bottom).}
    \label{fig:2014_divergence}
\end{figure}

\textbf{2. Regional Outliers: Lebanon and the Virgin Islands}
We also identified outliers that defy their regional or income group trends (Figure \ref{fig:2014_outliers}).
\begin{itemize}
    \item \textbf{Lebanon:} Like Azerbaijan, it punches above its weight, showing a positive Internet deviation despite average GDP.
    \item \textbf{Virgin Islands (U.S.):} A rare "High Income" failure. Despite being in the Green cluster, its internet access lags significantly, likely due to island infrastructure challenges.
\end{itemize}

\begin{figure}[h]
    \centering
    % Row 1: Lebanon
    \begin{subfigure}[b]{0.45\textwidth}
        \centering
        \includegraphics[width=\textwidth]{Vis_4/Results/Lebanon_label.png}
        \caption{Lebanon: Over-Performer}
    \end{subfigure}
    \hfill
    \begin{subfigure}[b]{0.45\textwidth}
        \centering
        \includegraphics[width=\textwidth]{Vis_4/Results/Lebanon_bar.png}
        \caption{Positive Internet Z-Score}
    \end{subfigure}
    
    \vspace{1em}
    
    % Row 2: Virgin Islands
    \begin{subfigure}[b]{0.45\textwidth}
        \centering
        \includegraphics[width=\textwidth]{Vis_4/Results/Virgin_label.png}
        \caption{Virgin Islands: Under-Performer}
    \end{subfigure}
    \hfill
    \begin{subfigure}[b]{0.45\textwidth}
        \centering
        \includegraphics[width=\textwidth]{Vis_4/Results/Virgin_bar.png}
        \caption{Negative Internet Z-Score (Green Cluster)}
    \end{subfigure}
    \caption{Regional Anomalies (2014).}
    \label{fig:2014_outliers}
\end{figure}

\subsubsection{The 2018 Analysis: Maturation and Mismatches}
By 2018 (Figure \ref{fig:2018_general}), the global trend tightens, but extreme outliers emerge that completely decouple digital growth from economic constraints.

\begin{figure}[H]
    \centering
    \includegraphics[width=1.0\textwidth]{Vis_4/Results/2018_General.png}
    \caption{The 2018 Global Landscape. Kosovo appears as the highest outlier in the purple cluster, nearing the ceiling of 100\% access.}
    \label{fig:2018_general}
\end{figure}

\textbf{3. The Balkan Miracle: Kosovo}
Kosovo represents the ultimate "Over-Performer" (Figure \ref{fig:2018_kosovo}). Its Z-score profile is the most unique in the dataset: a GDP Z-score that is negative (below average) paired with an Internet Z-score near +1.0 (exceptional). This proves that digital infrastructure can be fully decoupled from national wealth.

\begin{figure}[h]
    \centering
    \begin{subfigure}[b]{0.45\textwidth}
        \centering
        \includegraphics[width=\textwidth]{Vis_4/Results/Kosovo_label.png}
        \caption{Kosovo Tooltip}
    \end{subfigure}
    \hfill
    \begin{subfigure}[b]{0.45\textwidth}
        \centering
        \includegraphics[width=\textwidth]{Vis_4/Results/Kosovo_bar.png}
        \caption{The "Decoupled" Profile}
    \end{subfigure}
    \caption{Kosovo (2018): High connectivity despite low economic output.}
    \label{fig:2018_kosovo}
\end{figure}

\textbf{4. Classification Mismatches and Scale}
Finally, the tool exposed limitations in official labels (Figure \ref{fig:2018_mismatches}).
\begin{itemize}
    \item \textbf{Kyrgyz Republic:} The tooltip confirms it is officially "Lower Middle Income," but our algorithm upgrades it to "Medium Development" due to its high connectivity.
    \item \textbf{Luxembourg:} Validates the Log-Scale design. It is an outlier not due to a mismatch, but purely due to the extreme magnitude of its GDP.
\end{itemize}

\begin{figure}[h]
    \centering
    % Row 1: Kyrgyz
    \begin{subfigure}[b]{0.45\textwidth}
        \centering
        \includegraphics[width=\textwidth]{Vis_4/Results/Kyr_label.png}
        \caption{Kyrgyz Rep: Label Mismatch}
    \end{subfigure}
    \hfill
    \begin{subfigure}[b]{0.45\textwidth}
        \centering
        \includegraphics[width=\textwidth]{Vis_4/Results/Kyr_bar.png}
        \caption{Kyrgyz Rep: Z-Scores}
    \end{subfigure}
    
    \vspace{1em}
    
    % Row 2: Luxembourg
    \begin{subfigure}[b]{0.45\textwidth}
        \centering
        \includegraphics[width=\textwidth]{Vis_4/Results/Lux_label.png}
        \caption{Luxembourg: Scale Outlier}
    \end{subfigure}
    \hfill
    \begin{subfigure}[b]{0.45\textwidth}
        \centering
        \includegraphics[width=\textwidth]{Vis_4/Results/Lux_bar.png}
        \caption{Luxembourg: Extreme GDP}
    \end{subfigure}
    \caption{Structural Outliers (2018). Identifying mismatches in World Bank classifications and validating the logarithmic scale.}
    \label{fig:2018_mismatches}
\end{figure}

%%%%%%%%%%%%%%%%%%%%%%%%%%%%%%%%%%%%%%%%%%%%%%%%%%%%%%%%%%%%%
\newpage
\section{How is the global connected population distributed among regions and countries, reflecting the hierarchical structure?}

\subsection{Problem characterization}

\subsection{Data and task abstraction}

\subsection{Interaction and visual encoding}

\subsection{Algorithmic implementation}

\subsection{Results}

\newpage
\section{Instructions to run the app}

To ensure accessibility and ease of evaluation, the application has been deployed to the cloud. However, it can also be executed locally within the RStudio environment.

\subsection{Option 1: Online Access (Recommended)}
The most immediate way to interact with the tool is via the web deployment on ShinyApps.io. This version requires no local configuration or installation.

\begin{itemize}
    \item \textbf{URL:} \href{https://cmanzano.shinyapps.io/World_Bank_Project/}{https://cmanzano.shinyapps.io/World\_Bank\_Project/}
    \item \textbf{Status:} Active and publicly accessible.
\end{itemize}

\subsection{Option 2: Local Installation}
If you wish to run the source code locally, please follow these steps to set up the environment and dependencies.

\subsubsection*{Prerequisites}
\begin{itemize}
    \item \textbf{R:} Version 4.0.0 or higher.
    \item \textbf{RStudio:} Recommended IDE for managing the Shiny runtime.
\end{itemize}

\subsubsection*{Step 1: Install Dependencies}
The project relies on a comprehensive suite of R packages for reactivity, geospatial rendering, and data manipulation. We have provided a helper script \texttt{install\_dependencies.R} in the root directory. You can install all required libraries by running the following command in the R console:

\begin{verbatim}
source("install_dependencies.R")
\end{verbatim}

Alternatively, the dependencies can be manually installed. They are categorized as follows:

\begin{itemize}
    \item \textbf{App Framework:} \texttt{shiny}, \texttt{shinydashboard}
    \item \textbf{Tidyverse \& Data Manipulation:} \texttt{tidyverse} (including \texttt{dplyr}, \texttt{tidyr}, \texttt{readr}, \texttt{tibble}, \texttt{purrr}, \texttt{stringr}, \texttt{forcats}), and \texttt{lubridate} for date handling.
    \item \textbf{Visualization:} \texttt{ggplot2}, \texttt{plotly}, \texttt{scales}, \texttt{colorspace}.
    \item \textbf{Geospatial \& Mapping:} \texttt{sf}, \texttt{rnaturalearth}, \texttt{rnaturalearthdata}.
\end{itemize}

\subsubsection*{Step 2: Run the Application}
Once the dependencies are installed, open the \texttt{app.R} file in RStudio and click the \textbf{"Run App"} button located at the top right of the script editor, or execute the following command in the console:

\begin{verbatim}
shiny::runApp()
\end{verbatim}

\textit{Note: Ensure that the working directory is set to the project root so that relative paths to the modules and CSV data files are resolved correctly.}
\end{document}
